\documentclass[10pt,oneside]{book}

\ifx \HCode\Undef
\newcommand*{\topdir}{.}%
\else
\def\pgfsysdriver{pgfsys-tex4ht.def}
\newcommand*{\topdir}{..}%
\fi

\usepackage{calc}
\usepackage[capitalise]{cleveref}
\usepackage{enumitem}
\usepackage{environ}
\usepackage{graphicx}
\usepackage{hyperref}
\usepackage{lipsum}
\usepackage{makeidx}
\usepackage{minted}
\usepackage{nameref}
\usepackage{tikz}
\usepackage[tikz]{bclogo}
\usepackage{ulem}
\usepackage[many]{tcolorbox}
\usepackage{etoolbox}
\usepackage{ifthen}
\usepackage{fontawesome}
\usepackage{xcolor}
\usepackage{tabularx}
\usepackage[margin=1in]{geometry}
\usepackage[small,compact]{titlesec}
\usepackage[binary,squaren]{SIunits}
\usepackage[framemethod=tikz]{mdframed}
\usetikzlibrary{arrows}
\usetikzlibrary{calc}
\usetikzlibrary{decorations.markings}

\makeindex
\AtBeginDocument{\renewcommand{\bibname}{References}}

\newcommand{\ConsusVersion}{0.0.dev}
\newcommand{\code}[1]{\texttt{#1}}

% http://texblog.org/2012/03/21/cross-referencing-list-items/
\makeatletter
\def\namedlabel#1#2{\begingroup
    #2%
    \def\@currentlabel{#2}%
    \phantomsection\label{#1}\endgroup
}
\makeatother

\setcounter{secnumdepth}{3}
\setcounter{tocdepth}{2}

\title{Consus Reference Manual v\ConsusVersion}
\author{Consus Team}

\BeforeBeginEnvironment{minted}{\begin{tcolorbox}}%
\AfterEndEnvironment{minted}{\end{tcolorbox}}%
\newminted{c}{samepage}
\newminted{console}{samepage}
\newminted{go}{samepage}
\newminted{java}{samepage}
\newminted{javascript}{samepage}
\newminted{json}{samepage}
\newminted{pycon}{samepage}
\newminted{python}{samepage}
\newminted{ruby}{samepage}

% These boxes are all inspired by a solution from here:
% http://tex.stackexchange.com/questions/66820/how-to-create-highlight-boxes-in-latex

\NewEnviron{consuscaution}[1]
  {\par\medskip\noindent
  \begin{tikzpicture}
    \node[inner sep=0pt] (box) {\parbox[t]{.99\textwidth}{%
      \begin{minipage}{.2\textwidth}
      \centering\tikz[scale=5]\node[scale=2]{\bctakecare};
      \end{minipage}%
      \begin{minipage}{.75\textwidth}
      \textbf{#1}\par\smallskip
      \BODY
      \end{minipage}\hfill}%
    };
    \draw[red!75!black,line width=3pt] 
      ( $ (box.north east) + (-5pt,3pt) $ ) -- ( $ (box.north east) + (0,3pt) $ ) -- ( $ (box.south east) + (0,-3pt) $ ) -- + (-5pt,0);
    \draw[red!75!black,line width=3pt] 
      ( $ (box.north west) + (5pt,3pt) $ ) -- ( $ (box.north west) + (0,3pt) $ ) -- ( $ (box.south west) + (0,-3pt) $ ) -- + (5pt,0);
  \end{tikzpicture}\par\medskip%
}

% These boxes are all inspired by a solution from here:
% http://tex.stackexchange.com/questions/304449/combine-minted-and-tcolorbox-for-code-from-file-inputminted

\newcounter{codeCount}[chapter]
\crefname{codeCount}{Code Listing}{Code Listings}

\tcbuselibrary{listings}
\tcbuselibrary{minted}

\newcolumntype{\CeX}{>{\centering\let\newline\\\arraybackslash}X}%
\newcommand{\SymbolAndText}[2]{%
  \begin{tabularx}{\textwidth}{c\CeX c}%
    #1 & #2
  \end{tabularx}%
}

\newtcbinputlisting[use counter=codeCount, number format=\arabic]{\codeFromFile}[4]{%
  listing engine=minted,
  minted language=#1,
  listing file={#2},
  minted options={autogobble,linenos,breaklines},
  listing only,
  size=title,
  arc=1.5mm,
  breakable,
  enhanced jigsaw,
  colframe=red!75!black,
  coltitle=white,
  boxrule=0.5mm,
  colback=white,
  coltext=black,
  title=\SymbolAndText{\faCode}{%
    \textbf{Code Listing \thetcbcounter}\ifthenelse{\equal{#3}{}}{}{\textbf{:} #3}%
  },
  label=code:#4
}

\newtcblisting[use counter=codeCount, number format=\arabic]{codeBox}[3]{
  listing engine=minted,
  minted language=#1,
  minted options={autogobble,linenos,breaklines},
  listing only,
  size=title,
  arc=1.5mm,
  breakable,
  enhanced jigsaw,
  colframe=red!75!black,
  coltitle=white,
  boxrule=0.5mm,
  colback=white,
  coltext=black,
  title=\SymbolAndText{\faCode}{%
    \textbf{Code Listing \thetcbcounter}\ifthenelse{\equal{#2}{}}{}{\textbf{:} #2}%
  },
  label=code:#3
}


\begin{document}

\frontmatter
\maketitle
\tableofcontents

\mainmatter

\part{Working with Consus}

\chapter{Installing Consus}
\label{chap:installation}

Consus provides two simple installation methods that work for a variety of
environments.  For most users, the 64-bit Linux binary will be the fastest and
easiest way to install Consus.  Users on other platforms, or those looking to
customize Consus, can use the provided tarballs to install on their platform of
choice.  \cref{tab:compat} provides a summary of the different installation
methods to help decide which method best suits your needs.

\begin{table}[h!]
\centering
\begin{tabular}{r|l|l}
            &   64-bit Linux Binary & Source Tarballs \\
64-bit Linux compatible  & \faCheck  & \faCheck \\
Tested binaries          & \faCheck  & \faTimes \\
Client-only installation & \faTimes  & \faCheck \\
Other operating systems  & \faTimes  & \faCheck \\
Other CPU architectures  & \faTimes  & \faTimes\footnotemark \\
\end{tabular}
\caption{Comparison of Consus installation methods.}
\label{tab:compat}
\end{table}

\footnotetext{Support for other architectures may be added as they become more
common in server class hardware.}

\section{Binaries for 64-bit Linux Platforms}
\label{sec:install:bin}

A tarball containing the binaries suitable for running on most modern Linux
distributions can be found on the \href{http://consus.io/download/}{Consus
downloads page}.  The binaries are compiled to work on most modern Linux
distributions---specifically those using Linux kernel 2.6.19 or later and glibc
2.6 or later\footnotemark.  The commands to extract the binaries from the
tarball and add the binaries to your path are shown in
\cref{code:install:bin-global}.

\footnotetext{Linux 2.6.19 was released in November, 2006 and GNU libc 2.6.1 was
released in August, 2007.  Ubuntu 7.10, Fedora 7, Debian 5, and Centos 6 are the
first versions of major Linux distributions to meet the minimum software
requirements.}

\codeFromFile{console}{\topdir/install/linux-amd64}{Install to \code{/usr/local/consus}}{install:bin-global}

This will create the directory \code{/usr/local/consus} and populate it with all
the binaries necessary to run a Consus cluster.  While it is convention to
install files into \code{/usr/local}, the installation process is largely
directory-agnostic and can be installed anywhere on the filesystem.  This
portability is useful for installations where root access is not available and
you would like to install in your home directory.  \cref{code:install:bin-home}
shows how to install Consus into the directory \code{\$HOME/consus}.

\codeFromFile{console}{\topdir/install/linux-amd64-home}{Install to \code{\$HOME/consus}}{install:bin-home}

\begin{consuscaution}{Compatibility Across Linux Distributions}
    The Consus team has taken great care to test this binary on a variety of
    mainstream Linux distributions.  Custom distributions where glibc or the
    kernel have been built with atypical options, or where everything is
    statically linked might not be able to run Consus properly.  If the binaries
    do not work for you and your distribution is heavily customized, or built
    from source, consider following the source-based installation instructions
    instead as described in \cref{sec:install:src}.

    \phantom{p}
    
    Feedback about which platforms work well and which have problems is always
    appreciated as the team continues to improve Consus.
\end{consuscaution}

\section{Installing from Source Tarballs}
\label{sec:install:src}

Installing Consus from source tarballs is more complicated than the binary
installation because it is dependent upon your platform and requires a few more
commands to complete the installation.  The upside of the source tarballs is
that they work in places that the pre-built binaries do not, and can be used to
build distribution-specific packages.

The installation broadly divides into two distinct steps.  In the first step, we
will walk through preparing your system with all the prerequisites necessary to
build Consus.  This includes setting up a working C compiler, and installing
dependencies not managed by the Consus team.  In the second step, we will walk
through the steps to install the packages maintained by the Consus team that
together comprise Consus.

\subsection{Preparing for Installation}

The individual preparation steps depend heavily upon the target of your Consus
installation.  We will walk through the exact commands necessary to install
Consus on some of the common Linux distributions.  Depending on your setup, you
may need to perform different or slightly altered steps for installing Consus.

% XXX
% \paragraph{Preparing Debian}
% \paragraph{Preparing Fedora}
% \paragraph{Preparing CentOS}

\subsubsection*{Preparing Ubuntu}

On recent versions of Ubuntu, all Consus' dependencies can be installed directly
from packages.  \cref{code:prereq:src:ubuntu14.04} shows the instructions
starting with Ubuntu 14.04 and continuing through 16.10.

\codeFromFile{console}{\topdir/install/ubuntu14.04-src-prereqs}%
{Prepare Installation on Ubuntu 14.04 and later}{prereq:src:ubuntu14.04}

Installing on Ubuntu 12.04 requires a bit more work as the Google logging
library is not packaged and the packaged LevelDB version is too old for Consus.
\cref{code:prereq:src:ubuntu12.04} shows the complete steps for installing
prerequisite packages and for installing the other dependencies from source.

\codeFromFile{console}{\topdir/install/ubuntu12.04-src-prereqs}%
{Prepare Installation on Ubuntu 12.04}{prereq:src:ubuntu12.04}

\subsection{Installing Consus Components}

With the prerequisites to Consus installed, we can install the rest of Consus in
a simple loop.  If we download all tarballs from the Consus Download Site % XXX
we can then install them by repeating the configure-make-install process for
each.

\codeFromFile{console}{\topdir/install/source}{Install to \code{/usr/local}}{install:src}


%\part{Developing Consus}

%\part{API Reference}

\ifx \HCode\Undef % >>>>>>>>>>>>>>>>>>>
\backmatter

% Index
\clearpage
\addcontentsline{toc}{chapter}{Index}
\printindex

% References
\clearpage
\addcontentsline{toc}{chapter}{References}
\bibliographystyle{plain}
\bibliography{consus}
\fi               % <<<<<<<<<<<<<<<<<<<

\end{document}
